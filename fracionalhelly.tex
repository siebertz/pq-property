\documentclass[11pt, fleqn]{article}
\usepackage[english]{babel}

\usepackage[lmargin=1.1in,rmargin=1.1in,bottom=1.3in,top=1.3in,
twoside=False]{geometry}

\usepackage{relsize,xspace}
 \usepackage{xcolor}
 \usepackage{mathtools}
 \usepackage{todonotes}
 \usepackage{comment}
\usepackage{microtype}
\usepackage{amsmath}
\usepackage{amssymb}
\usepackage{amsfonts}
\usepackage{stmaryrd}
\usepackage{bm}
\usepackage{tikz}
\usepackage{refcount}
\usepackage{wrapfig}

\usepackage{marginnote}


\definecolor{blue}{rgb}{0.1,0.2,0.5}
\definecolor{brown}{rgb}{0.6,0.6,0.2}
\usepackage[ocgcolorlinks, linkcolor={blue}, citecolor={brown}]{hyperref}

\usepackage[amsmath,thmmarks,hyperref]{ntheorem}
\usepackage{cleveref}


\crefformat{page}{#2page~#1#3}%
\Crefformat{page}{#2Page~#1#3}%
\crefformat{equation}{#2(#1)#3}%
\Crefformat{equation}{#2(#1)#3}%
\crefformat{figure}{#2Figure~#1#3}%
\Crefformat{figure}{#2Figure~#1#3}%
\crefformat{section}{#2Section~#1#3}
\Crefformat{section}{#2Section~#1#3}
\crefformat{chapter}{#2Chapter~#1#3}
\Crefformat{chapter}{#2Chapter~#1#3}
\crefformat{chapter*}{#2Chapter~#1#3}
\Crefformat{chapter*}{#2Chapter~#1#3}
\crefformat{part}{#2Part~#1#3}
\Crefformat{part}{#2Part~#1#3}
\crefformat{enumi}{#2(#1)#3}
\Crefformat{enumi}{#2(#1)#3}

\usepackage{enumerate}

\usepackage{latexsym}

% BEGIN ntheorem configuration

\theoremnumbering{arabic}
\theoremstyle{plain}
\theoremsymbol{}
\theorembodyfont{\itshape}
\theoremheaderfont{\normalfont\bfseries}
\theoremseparator{.}

\newtheorem{theorem}{Theorem}
\crefformat{theorem}{#2Theorem~#1#3}
\Crefformat{theorem}{#2Theorem~#1#3}
\renewcommand{\setminus}{-}
\newcommand{\newtheoremwithcrefformat}[2]{%
  \newtheorem{#1}[lemma]{#2}%
  \crefformat{#1}{##2\MakeUppercase#1~##1##3}%
  \Crefformat{#1}{##2\MakeUppercase#1~##1##3}%
}
\newcommand{\newseptheoremwithcrefformat}[2]{%
  \newtheorem{#1}{#2}%
  \crefformat{#1}{##2\MakeUppercase#1~##1##3}%
  \Crefformat{#1}{##2\MakeUppercase#1~##1##3}%
}

\newseptheoremwithcrefformat{lemma}{Lemma}
\newtheoremwithcrefformat{proposition}{Proposition}
\newtheoremwithcrefformat{observation}{Observation}
\newtheoremwithcrefformat{conjecture}{Conjecture}
\newtheoremwithcrefformat{corollary}{Corollary}
\newseptheoremwithcrefformat{claim}{Claim}
\theorembodyfont{\upshape}
\newtheoremwithcrefformat{example}{Example}
\newtheoremwithcrefformat{remark}{Remark}
\newseptheoremwithcrefformat{definition}{Definition}

\theoremstyle{nonumberplain}
\theoremheaderfont{\scshape}
\theorembodyfont{\normalfont}
\theoremsymbol{\ensuremath{\square}}
\newtheorem{proof}{Proof}

\theoremsymbol{\ensuremath{\lrcorner}}
\newtheorem{clproof}{Proof}

% END ntheorem configuration

\newcommand{\set}[1]{\{#1\}}
\renewcommand{\subset}{\subseteq}

%\setlength{\parskip}{0.1cm}
%\setlength{\parindent}{0cm}
%\setlength{\mathindent}{1cm}

\newcommand{\wcol}{\mathrm{wcol}}
\newcommand{\col}{\mathrm{col}}
\newcommand{\adm}{\mathrm{adm}}
\newcommand{\tw}{\mathrm{tw}}
\newcommand{\WReach}{\mathrm{WReach}}
\newcommand{\SReach}{\mathrm{SReach}}
\newcommand{\wcolorder}{\sqsubseteq}
\newcommand{\Oof}{\mathcal{O}}
\newcommand{\CCC}{\mathcal{C}}
\newcommand{\NNN}{\mathcal{N}}
\newcommand{\WWW}{\mathcal{W}}
\newcommand{\DDD}{\mathcal{D}}
\newcommand{\PPP}{\mathcal{P}}
\newcommand{\FFF}{\mathcal{F}}
\newcommand{\GGG}{\mathcal{G}}
\newcommand{\YYY}{\mathcal{Y}}
\newcommand{\nei}{\mathrm{nei}}
\renewcommand{\ker}{\mathrm{ker}}
\newcommand{\core}{\mathrm{core}}

\newcommand{\cutrk}{\mathrm{cutrk}}
\newcommand{\rank}{\mathrm{rank}}
\newcommand{\rw}{\mathrm{rw}}


\newcommand{\grad}{\nabla}
\newcommand{\ds}{\mathbf{ds}}
\newcommand{\cl}{\mathrm{cl}}
\newcommand{\cst}{\alpha}

\newcommand{\fnei}{f_{\nei}}
\newcommand{\fwcol}{f_{\wcol}}
\newcommand{\fker}{f_{\ker}}
\newcommand{\fproj}{f_{\mathrm{proj}}}
\newcommand{\fcl}{f_{\cl}}
\newcommand{\fgrad}{f_{\grad}}
\newcommand{\fpaths}{f_{\mathrm{pth}}}
\newcommand{\fapx}{f_{\mathrm{apx}}}
\newcommand{\fcore}{f_{\mathrm{core}}}
\newcommand{\ffin}{f_{\mathrm{fin}}}

\newcommand\blfootnote[1]{%
  \begingroup
  \renewcommand\thefootnote{}\footnote{#1}%
  \addtocounter{footnote}{-1}%
  \endgroup
}

\newcommand{\suchthat}{ \colon }
\newcommand{\sth}{ \colon }
\newcommand{\ie}{i.e.\@ }
\newcommand{\N}{\mathbb{N}}
\newcommand{\R}{\mathbb{R}}
\newcommand{\tup}[1]{\bar{#1}}
\renewcommand{\phi}{\varphi}
\renewcommand{\epsilon}{\varepsilon}
\newcommand{\str}{\mathfrak}
\newcommand{\strA}{\str{A}}
\newcommand{\strB}{\str{B}}
\newcommand{\FO}{\mathrm{FO}}
\newcommand{\minor}{\preccurlyeq}
\newcommand{\dist}{\mathrm{dist}}
\newcommand{\indx}{\mathrm{index}}
\renewcommand{\mid}{~:~}

\newcommand{\profnum}{\widehat{\nu}}
\newcommand{\projnum}{\mu}
\newcommand{\projprof}{\widehat{\mu}}

\newcommand{\abs}[1]{\ensuremath{\left\lvert#1\right\rvert}}

\newcommand{\im}{\mathrm{im}}
\newcommand{\rg}{\mathrm{rg}}
\newcommand{\from}{\colon}

\title{On the Hadwiger–Debrunner (p, q)-property and duality between
packing and domination on nowhere dense graph classes}
\author{Michał Pilipczuk \and Sebastian Siebertz \and Szymon Toruńczyk}

\begin{document}
\maketitle

\begin{abstract}
\noindent 
A family of sets $\FFF$ satisfies the Hadwiger-Debrunner $(p, q)$-property if among every 
$p$ sets $F_1,\ldots, F_p$ of $\FFF$, there is $I\subseteq \{1,\ldots, p\}$ of order at least $q$ such that $\bigcap_{i\in I}F_i\neq \emptyset$. We prove that a monotone class $\CCC$ of graphs is nowhere 
dense if and only if for every $r\in \N$ and $q\in \N$ there exists 
$p\in N$ such that for every $G\in\CCC$, every set system
$\NNN_r$ consisting of the $r$-neighborhoods of vertices of
pairwise distance at most $2r$ has the $(p,q)$-property. 

A transversal set (or hitting set) of $\FFF$ is a set of elements intersecting each $F\in \FFF$. The transversality $\tau$ of
$\FFF$ is the minimum size of a transversal set. The packing 
number $\nu$ of $\FFF$ is the maximum number of
pairwise disjoint sets in $\FFF$.
Using the $(p,q)$-property, Bousquet and Thomass\'e
proved that there exists a function $f$ such that, for every 
$r\in \N$, every graph~$G$ of distance-$r$ VC-dimension $d$ 
satisfies $\tau_r\leq f(\nu_r,d)$, where $\tau_r$ is
the transversality of the set of $r$-neighborhoods of
vertices of $G$ and $\nu_r$ is its packing number. We prove
that if $K_t\not\minor G$, then $G$ has distance-$r$ 
VC-dimension at most $t-1$. 
We then prove the
dependence $\tau_r(G)\in \Oof(\nu_r^{1+\epsilon})$
for every graph of a nowhere dense class and 
every $\epsilon>0$. We prove that this bound is
tight, by giving an example of a nowhere dense
class such that $\tau_r(G)\in \Omega(\nu_r^{1+\epsilon})$. 
\end{abstract}

\section{The Hadwiger-Debrunner (p,q)-property}

A family of sets $\FFF$ satisfies the Hadwiger-Debrunner 
$(p, q)$-property if among every 
$p$ sets $F_1,\ldots, F_p$ of $\FFF$, there is $I\subseteq \{1,\ldots, p\}$ of order at least $q$ such that $\bigcap_{i\in I}F_i\neq \emptyset$. 

\begin{theorem}
Let $\CCC$ be a monotone class of graphs. Then $\CCC$ 
is nowhere 
dense if and only if for every $r\in \N$ and $q\in \N$ there exists 
$p\in N$ such that for every $G\in\CCC$, every set system
$\NNN_r$ consisting of the $r$-neighborhoods of vertices of
pairwise distance at most $2r$ has the $(p,q)$-property. 
\end{theorem}
\begin{proof}
If $\CCC$ is nowhere dense, then it is uniformly quasi-wide, 
say with margins $s:\N\rightarrow\N$ and $N:\N\times\N\rightarrow \N$. 
Let $r\in \N$ and $q\in \N$ and let $p=N(q\cdot s(r),r)$. We claim that for every graph $G\in \CCC$, every set system
$\NNN_r$ consisting of the $r$-neighborhoods of vertices of
pairwise distance at most $2r$ has the $(p,q)$-property. 

Let $A=\{v_1,\ldots, v_k\}\subseteq V(G)$ with pairwise distance
at most $2r$, that is, $N_r(v_i)\cap N_r(v_j)\neq \emptyset$
for all $1\leq i,j\leq k$. As $\CCC$ is uniformly quasi-wide, 
we can delete from $G$ a set $S$ of at most $s(r)$ vertices
such that there is a subset $B\subseteq A\setminus S$ of 
size at least $q\cdot s(r)$ which is $r$-scattered. As $N_r(v_i)\cap N_r(v_j)\neq \emptyset$ in $G$, we have $S\cap N_r(v_i)\neq\emptyset$ for all $1\leq i\leq k$. Hence, there must be $s\in S$ with $s\in N_r(v)$ for at least $q$ elements $v\in B$.

Conversely, the $2r$-subdivisions of complete graphs do not 
have the $(p,3)$-property for any $p\in \N$. 
\end{proof}

\section{Distance VC-dimension}

The $2$VC-dimension
of a graph is the largest set which has a neighbour for each 
subset of size $2$. Obviously, the $2$VC-dimension of $G$
bounds its VC-dimension. 

\begin{theorem}
If $K_t\not\minor_rG$, then 
$G^r$ has $2$VC-dimension at most $t-1$. 
\end{theorem}

\begin{proof}
Assume there is a set $A=\{a_1,\ldots, a_t\}$ of size $t$ such that
for all subsets $\{i,j\}\subseteq \{1,\ldots,t\}$ of size $2$ 
there is an vertex $v_{ij}$ with 
$N_r[v_{ij}]\cap A=\{a_i,a_j\}$.
For each subset $\{i,j\}\subseteq \{1,\ldots,t\}$ of size $2$, choose a vertex $u_{ij}$ so that:
\begin{enumerate}[(1)]
\item\label{p:i} $\dist(v_{ij},u_{ij})+\dist(u_{ij},a_i)\leq r$;
\item\label{p:j} $\dist(v_{ij},u_{ij})+\dist(u_{ij},a_j)\leq r$; and
\item\label{p:min} subject to conditions \eqref{p:i} and \eqref{p:j}, $\max(\dist(u_{ij},a_i),\dist(u_{ij},a_j))$ is minimized.
\end{enumerate}
Observe that $u_{ij}$ is well-defined since setting $u_{ij}=v_{ij}$ satisfies the first two conditions.

Let $P^i_{ij}$ and $P^j_{ij}$ be arbitrarily chosen shortest paths between $u_{ij}$ and~$a_i$, and between $u_{ij}$ and~$a_j$, respectively.
We now establish some basic properties of paths $P^i_{ij}$ and $P^j_{ij}$ following from the choice of $u_{ij}$.

\begin{claim}\label{cl:ineq}
For each vertex $x$ on $P^i_{ij}$ we have $\dist(v_{ij},x)+\dist(x,a_i)\leq r$, and
for each vertex $y$ on $P^j_{ij}$ we have $\dist(v_{ij},y)+\dist(y,a_j)\leq r$.
\end{claim}
\begin{clproof}
We prove only the first statement for the second is symmetric.
We have
$$\dist(v_{ij},x)+\dist(x,a_{i})\leq \dist(v_{ij},u_{ij})+\dist(u_{ij},x)+\dist(x,a_{i})=\dist(v_{ij},u_{ij})+\dist(u_{ij},a_{i})\leq r,$$
where the last equality is due to $x$ lying on a shortest path between $u_{ij}$ and $a_i$, and the last inequality is by condition~\eqref{p:i}.
\end{clproof}

\begin{claim}\label{cl:closer}
Suppose $x$ is a vertex on $P^i_{ij}$ that is different from $u_{ij}$. Then $\dist(x,a_i)<\dist(x,a_j)$.
Symmetrically, if $y$ lies on $P^j_{ij}$ and is different from $u_{ij}$, then $\dist(y,a_i)>\dist(y,a_j)$.
Consequently, paths $P^i_{ij}$ and $P^j_{ij}$ share only one vertex, being the endpoint $u_{ij}$.
\end{claim}
\begin{clproof}
We prove only the first claim, for the second is symmetric and the third directly follows from the first two.
Suppose for contradiction that $\dist(x,a_i)\geq \dist(x,a_j)$.
By \cref{cl:ineq} we have 
$$\dist(v_{ij},x)+\dist(x,a_i)\leq r.$$
On the other hand, since $\dist(x,a_i)\geq \dist(x,a_j)$, we have
$$\dist(v_{ij},x)+\dist(x,a_j)\leq\dist(v_{ij},x)+\dist(x,a_i)\leq r.$$
We conclude that $x$ satisfies conditions \eqref{p:i} and \eqref{p:j} from the definition of $u_{ij}$.
However, since $x\neq u_{ij}$ and $x$ lies on a shortest path between $u_{ij}$ and $a_i$, we have $\dist(x,a_i)<\dist(u_{ij},a_i)$.
Therefore,
$$\dist(x,a_j)\leq \dist(x,a_i)<\dist(u_{ij},a_i)\leq \max(\dist(u_{ij},a_i),\dist(u_{ij},a_j)).$$
Thus, the existence of $x$ contradicts condition \eqref{p:min} from the definition of $u_{ij}$.
\end{clproof}

Now, define paths $Q^i_{ij}$ and $Q^j_{ij}$ as follows:
\begin{itemize}
\item if $\dist(u_{ij},a_i)<\dist(u_{ij},a_j)$, then $Q^{i}_{ij}=P^{i}_{ij}$ and $Q^{j}_{ij}=P^{j}_{ij} - \{u_{ij}\}$;
\item if $\dist(u_{ij},a_i)>\dist(u_{ij},a_j)$, then $Q^{i}_{ij}=P^{i}_{ij} - \{u_{ij}\}$ and $Q^{j}_{ij}=P^{j}_{ij}$;
\item if $\dist(u_{ij},a_i)=\dist(u_{ij},a_j)$, then define $Q^i_{ij}$ and $Q^j_{ij}$ using any of the above.
\end{itemize}
Thus, by \cref{cl:closer} we have that paths $Q^{i}_{ij}$ and $Q^{j}_{ij}$ are disjoint. Moreover, for each vertex $x$ on $Q^{i}_{ij}$ we have $\dist(x,a_i)\leq \dist(x,a_j)$, and for each
vertex $y$ on $Q^{j}_{ij}$ we have $\dist(y,a_i)\geq \dist(y,a_j)$.

\begin{claim}\label{cl:intersect}
Let $\{i,j\}$ and $\{i',j'\}$ be two different subsets of size $2$ of $\{1,\ldots,t\}$.
Suppose that paths $Q^i_{ij}$ and $Q^{i'}_{i'j'}$ intersect.
Then $i=i'$.
\end{claim}
\begin{clproof}
Let $x$ be a vertex lying both on $Q^i_{ij}$ and $Q^{i'}_{i'j'}$. We first consider the corner case when $x=u_{ij}$.
Suppose first that $\dist(v_{ij},x)\geq \dist(v_{i'j'},x)$. Then by \cref{cl:ineq} we have
$$\dist(v_{i'j'},a_i)\leq \dist(v_{i'j'},x)+\dist(x,a_i)\leq \dist(v_{ij},x)+\dist(x,a_i)\leq r,$$
and analogously $\dist(v_{i'j'},a_{j})\leq r$. However, we assumed that $a_{i'}$ and $a_{j'}$ are the only vertices of~$A$ that are at distance at most $r$ from $v_{i'j'}$, hence $\{i,j\}=\{i',j'\}$,
a contradiction. Suppose then that $\dist(v_{ij},x)<\dist(v_{i'j'},x)$. 
Then we have
$$\dist(v_{ij},a_{i'})\leq \dist(v_{ij},x)+\dist(x,a_{i'})<\dist(v_{i'j'},x)+\dist(x,a_{i'})\leq r,$$
where the last equality follows from \cref{cl:ineq}.
Since $a_i$ and $a_j$ are the only vertices of $A$ that are at distance at most $r$ from $v_{ij}$, we infer that $i'\in \{i,j\}$. 
If $i'=i$ then we would be done, so suppose $i'=j$.
Since $x=u_{ij}$ and $x$ lies on $Q^i_{ij}$, by the definition of $Q^i_{ij}$ we have that $\dist(x,a_i)\leq \dist(x,a_j)=\dist(x,a_{i'})$. Therefore,
$$\dist(v_{i'j'},a_i)\leq \dist(v_{i'j'},x)+\dist(x,a_{i})\leq \dist(v_{i'j'},x)+\dist(x,a_{i'})\leq r.$$
where the last inequality follows from \cref{cl:ineq}.
Again, we assumed that $a_{i'}$ and $a_{j'}$ are the only vertices of $A$ that are at distance at most $r$ from $v_{i'j'}$, so $i\in \{i',j'\}$. If $i=i'$ then we are done, and otherwise we have $i=j'$.
Together with $i'=j$ this implies $\{i,j\}=\{i',j'\}$, a contradiction.

The second corner case when $x=u_{i'j'}$ leads to a contradiction in a symmetric manner.

We now move to the main case when $x\neq u_{ij}$ and $x\neq u_{i'j'}$.
Then by \cref{cl:closer} we have $\dist(x,a_i)<\dist(x,a_j)$ and $\dist(x,a_{i'})<\dist(x,a_{j'})$.
By symmetry, without loss of generality assume that $\dist(x,a_i)\leq \dist(x,a_{i'})$.
Observe now that
$$\dist(v_{i'j'},a_i)\leq \dist(v_{i'j'},x)+\dist(x,a_{i})\leq \dist(v_{i'j'},x)+\dist(x,a_{i'})\leq r,$$
where the last inequality follows from \cref{cl:ineq}.
Since we assumed that $a_{i'}$ and $a_{j'}$ are the only vertices of $A$ that are at distance at most $r$ from $v_{i'j'}$, we have $i\in \{i',j'\}$.
However, it cannot happen that $i=j'$, because $\dist(x,a_{i'})<\dist(x,a_{j'})$ and $\dist(x,a_{i'})\geq \dist(x,a_{i})$. We conclude that $i=i'$.
\end{clproof}

For each $i\in \{1,2,\ldots,t\}$ we define $X_i$ to be the union of vertex sets of paths $Q^i_{ij}$ for $j\neq i$.
Each of these paths has length at most $r$ and has $a_i$ as an endpoint, hence the subgraph induced by $X_i$ is connected and has radius at most $r$.
By \cref{cl:intersect}, sets $X_i$ are pairwise disjoint. Finally, observe that for each $\{i,j\}\subseteq \{1,\ldots,t\}$ with $i\neq j$, there is an edge between a vertex of $Q^{i}_{ij}$ and a vertex of $Q^{j}_{ij}$.
We conclude that $(X_i)_{i=1,\ldots,t}$ is a depth-$r$ minor model of $K_t$ in $G$, a contradiction.
\end{proof}

\section{On the duality between packing and domination in nowhere dense classes}

Dvo\v{r}\'ak proved that $\tau_r\leq \wcol_r(G)^2\nu_r$ for all $r\in \N$. However, $\wcol_r(G)$ can be as large as $n^\epsilon$ in an 
$n$-vertex graph from a nowhere dense class. However, we proved that we can compute a subgraph $G'$ of $G$ of order at most $\Oof(\tau_r^{1+\epsilon})$ and a set $Z\subseteq V(G')$ such that 
the smallest $r$-dominating set for $Z$ in $G'$ has size exactly $\tau_r$. Furthermore, we observed earlier, that the relation $\tau_r\leq \wcol_r(G)^2\nu_r$ holds also set-wise, that is, the smallest $r$-dominating set for $Z$ is at most as large as a $2r$-independent subset of $Z$ multiplied by $\wcol_r(G)^2$. Now, as $G'$ is small, 
we have $\wcol_r(G')\in \Oof(\tau_r^{1+\epsilon})$. 



\bibliographystyle{abbrv}
\bibliography{ref} 

\end{document}